\documentclass[12pt]{article}
\usepackage[pdftex]{hyperref}
\begin{document}

\section {Introduction}
\begin{itemize}
\item \textbf{Training Set} : is used to tune the \textbf{parameters} of an adaptive model
	\begin{description}
		\item What about adaptive model??
		\item How to design or select an good adaptive model??
	\end{description}
\item \textbf{Learning phase} : The precise form of the function(adaptive model) is determined during the \emph{training phase}

\item \textbf{General Workflow}
	\begin{enumerate}
		\item Preprocess:
			\begin{itemize}
				\item \textbf{Normalization} : For example : transforming different size of pics to the same one.
				\item \textbf{Reduction} : Reduce dimension or quantity, to speed up comuputation. How to reduce dimension/quantity while preserving the information in those data is not easy.
			\end{itemize}
		\item Training
		\item Test : We can't train our model on all data that we have. We have to separate some of them to verify our model.
		\item Output result
	\end{enumerate}
\end{itemize}

\textbf{Classification:}
\begin{itemize}
	\item \textbf{supervised learning} : Training data comprise examples of the input vectors \underline{along with} their corresponding target vectors
		\begin{itemize}
			\item \textbf{classification}:decrete
			\item \textbf{regression}:continuous
		\end{itemize}
	\item \textbf{unsupervised learning} : Training data consists a set of input vector \textbf{x} \underline{without} any corresponding target value
		\begin{itemize}
			\item \textbf{cluster} : to discover group of similar examples
			\item \textbf{density estimation} : to determine the distribution of data within the input space.
			\item \textbf{visualization} : to project the data from a high-dimensional space down to two/three dimensions for the purpose of \emph{visualization}. Data visualization is a hot field now. Is this can be used to preprocess data set,as mention in \emph{Reduction}?
		\end{itemize}
	\item \textbf{reinforcement learning}: finding suitable actions to take in a given situation in order to maximize a reward.\\
		Make a balance of exploration and exploitation
		\begin{itemize}
			\item \textbf{exploration} : explore the unknow space.
			\item \textbf{exploitation} : make use of the actions that are known to yield a high reward.
		\end{itemize}
		Do you remember PSO ???
		\begin{equation}
		V(t+1) = w*V(t) + C_{1}*R_{1}*(P(t) - X(t)) + C_{2}*R_{2}*(G(t) - X(t))
		\end{equation}
		\begin{equation}
		X(t+1) = X(t) + V(t+1)
		\end{equation}
		It also need a balance between exploitation and exploration.But it has a global attraction when particle explore the unknown space. It is lucky...HA!
	\item \textbf{deep learning} : a new branch of machine learning. \href{http://en.wikipedia.org/wiki/Deep\_learning}{wikipedia}	
\end{itemize}

\section {Mathematica}
\textbf{Covariance}
\begin{equation}
cov(X,Y) = E((X-\mu)(Y-\nu))
\end{equation}
Why?Why covariance is define like this?We can see what it is defined in \href{http://mathworld.wolfram.com/Covariance.html{mathworld}}:
\begin{quote}
Covariance provides a measure of the strength of the correlation between two or more sets of random variates.
\end{quote}

\begin{enumerate}
	\item It is a measure of correlation. \textbf{correlation} means two or more factors are equal. They play the role. So they should be the \textbf{same form} in covariance equation.... 
	\item It is product of the difference between item and expectation. Product is a good choice. It can refect the relationship perfectly. 
\end{enumerate}

\begin{quote}
The variance of a random variable X is its second central moment, the expected value of the squared deviation from the mea 
\end{quote}
	\begin{equation}
	\mu = E[(X-\mu)^2]
	\end{equation}
\begin{equation}
	\sigma^2(X) = E[(X-\mu)(X-\mu)]
\end{equation}
The variance can also be thought of as the covariance of a random variable with itself:
\begin{equation}
	cov(X,Y) = E((X-\mu)(Y-\mu))
\end{equation}

\end{document}

